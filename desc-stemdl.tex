%!TEX root = paper.tex
\subsection{STEMDL ({\tt stemdl})}

State of the art Scanning Transmission Electron Microscopes (STEM)
produce focused electron beams with atomic dimensions, and allow 
capturing diffraction patterns arising from the interaction of incident
electrons with nano-scale material volumes. Backing out the local atomic
structure of said materials requires compute- and time-intensive
analyses of these diffraction patterns (known as convergent beam
electron diffraction or CBED). Traditional analyses of CBED requires
iterative numerical solutions of partial differential equations and
comparison with experimental data to refine the starting material
configuration. This process is repeated anew for every newly acquired
experimental CBED pattern and/or probed material.



\begin{small}
\begin{table}
    \caption{Summary of the {\tt stemdl} benchmark.}\label{tab:stemdl-summary}
    \begin{tabular}{p{0.4\columnwidth}p{0.6\columnwidth}}
    \hline
    \hline
    {\bf Description}
    &
    Classification and reconstruction of convergent beam electron
    diffraction, CBED. \\
    \hline
    {\bf Objectives}
    &
    Classification for crystal space groups and reconstruction for local
    electron density using machine learning.
    \\
    \hline
    {\bf Challenge Stream} & Classification\\
    \hline
    {\bf Domain} & Solid-state Physics\\
    \hline
    {\bf Metrics} &  Classification accuracy and/or F1-score\\
    \hline
    % Score &  0.9 considered converged \\
    {\bf Data} 
        & Type: Images \\
        &  $[512\times512\times3]$, label: $[200]$ (Classification) \\
        &  $[256\times256\times256]$, label: $[256\times256]$ (Reconstruction) \\
        & Size: 548.7 GB for Classification\\
        & Training samples: 138.7K\\
        & Validation samples: 48.4\\
        & Reconstruction: 10 TB \\
        & Source: Oak Ridge National Laboratory (ORNL) \\
        & Location: OSTI Servers~\cite{laanait-scanning}\\
    \hline
    {\bf Reference Implementation} & AAIMS repository~\cite{stemdl-benchmark} \\
    & Model: ResNet-50 \\
    & Run Instructions: \cite{stemdl-benchmark} \\
    & Time-to-solution: 40 minutes on 60 V100 GPUs \\
    \hline
    {\bf References} & \cite{laanait-scanning,pan-20,laanait-ms-19} \\
    \hline
    \hline
    \end{tabular}
\end{table}
\end{small}


\noindent {\bf Benchmark Objectives and Metrics:} The scientific objective of the benchmark is to develop a universal classifier for space group of solid state materials, and  reconstruction of local electron density. As stated before, this is conventionally performed using expensive simulations. The goal here is to use explore the suitability of ML algorithms for performing advanced analysis of CBED. This benchmark aims to quantify this using a classification task. As such, the benchmark is set with the supervised learning focus where both the scientific metric is reflected by the classification accuracy of the ML model. The benchmark also desires to achieve better top-1 classification accuracy and/or F1-score compared to the reference implementation.

\smallskip

\noindent {\bf Data:} A single \href{https://doi.ccs.ornl.gov/ui/doi/70}{data sample}~\cite{laanait-scanning}
from this dataset is given by a three-dimensional array formed by stacking various CBED patterns simulated from the same material at different distinct material projections (i.e. crystallographic orientations). Each CBED pattern is a two-dimensional array with 32-bit floating-point image intensities. Associated with each data sample in the dataset is a host of material attributes or properties which are, in principle, retrievable via analysis of this CBED stack. The dataset has (1) 200 crystal space groups out of 230 unique mathematical discrete space groups and (2) local electron density which governs material's property. A more detailed description of the data can be found in CBED database~\cite{laanait-scanning}. The dataset is divided into three distinct sets, split across training (148,006 files), testing (18,749 files), and development (20,400 files). The distinct nature of these sets ensures that the model learns the generic symmetry based on space groups instead of memorizing a particular pattern for a material.

\smallskip

\noindent {\bf Reference Implementation:} A detailed description of the baseline implementation method can be found in~\cite{pan-20} and~\cite{laanait-ms-19} along with the reference
implementation deposited into the AAIMS repository~\cite{stemdl-benchmark}.



