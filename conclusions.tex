%!TEX root = paper.tex
\section{Conclusions}
\label{sec:conclusions}

In this paper, we have discussed the initiatives of the MLCommons Science Working Group for advancing the AI for Science through science-specific benchmarks. By collaboratively working with multiple communities, covering various international laboratories, academic institutes and industries, the working group has succeeded in identifying a number of key scientific problems, and developed benchmarks for them. While this is a notable step forward for AI benchmarking, it is significant step for AI benchmarking focused on science. The working group is also actively working on a number of future benchmarks, drawing expertise from various domains. These future benchmarks will cover additional domains, and will also include a variety of classes of ML algorithms, such as surrogate models, inference- and training-based evaluations, and generative models, to mention a few. The future work will also give emphasis to the FAIR aspects of the data, ensuring that all our datasets are FAIR compliant. The working group is aspiring to support submissions of evaluations, so that the community is aware of performance benefits of  different systems. 

We are very hopeful that this initiative becomes beneficial to the scientific community in a number of different ways, such as supporting easy selection of ML algorithms for a given scientific problem, or for pedagogical purposes. With such purposes, we are hopeful the combined effect of MLCommons is likely to make a significant difference in the AI community. 